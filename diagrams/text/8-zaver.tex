\chapter{Závěr}
\label{8-zaver}
Stavba dronu není lehký úkol, zvlášť pro někoho, kdo neumí elektrotechniku, ale pokud existuje nápad, jak dál rozvíjet geodezii, je potřeba se ho chytit.\\
Původně se práce měla věnovat nadstavbovým geodetickým úlohám nad dronem od firmy Microkopter. U zapůjčeného dronu nefungovala pouze radiová komunikace. Chtěl jsem vyměnit komunikační zařízení a s dronem začít létat, bohužel ovládací prvky se nedaly přeprogramovat a ani nefungovala komunikace prvků s počítačem. Z dronu byly odebrány kontroléry a zůstala kostra s motory a regulátory otáček. Po několika testováních jeden z regulátorů zkratoval a jelikož byl regulátor od firmy Microkopter drahý, byl nahrazen jiným (uvedený v komponentách).\\
Tím začala stavba dronu od nuly. Nové regulátory měly odlišný způsob komunikace a jiný způsob zapojení. Bohužel i zapůjčené baterie byly poškozené, z důvodu dlouhodobého nepoužívání.\\
Pro komunikaci kontroléru s regulátory byla prvně použita knihovna Servo. Většina projektů používá Ardunino Servo knihovnu pro komunikaci s regulátory otáček. Po pár neúspěšných testech letového kontroléru, jsem zjistil, kde spočívá problém. Knihovna Servo dokáže komunikovat s regulátory, ale pouze s frekvencí 50Hz. Pro let dronu musí být frekvence ovládání regulátorů větší než 100Hz. Proto jsem musel implementovat standartní PWM komunikační protokol pro regulátory. Komunikační protokol je závislý na době trvání výpočtu. S použitím platformy Arduino byla docílená frekvence ovládání regulátorů 250Hz.\\
S IMU jednotkou byly tež problémy. Při použití knihovny pro modul MPU9250, bylo čtení dat z modulu pomalé. Proto byl nastudován popis modulu a modul byl ovládán přes komunikaci I2C za použití registrů.\\
Pro ovládání byla vytvořena aplikace pro operační systém Android v programovacím jazyku Java. Jedná se o první aplikaci, kterou jsem kdy dělal. Na internetu je spousta návodů, podle kterých se dá naučit programovat aplikaci. Hodně mi pomohla dokumentace Android Developers. Bohužel se nepodařilo do aplikace implementovat obraz z kamery. Pro zprovoznění kamery bylo potřeba importovat knihovny v jiných programovacích jazycích a nastavit jejich sestavení. Nynější stav je, že kamera je připojená k aplikaci, ale nic nezobrazuje.\\
Jak bylo zmíněno na začátku, stavba dronu není lehký úkol. Proto se práce zabývá pouze stavbou a už ne geodetickými nadstavbami. S konstrukcí dronu bych chtěl pokračovat a zrealizovat nápady uvedené v úvodu.\\
Při konstrukci dronu byly zničeny čtyři desky Arduino UNO, šest regulátorů otáček, tři plastové vrtule a jedna IMU jednotka. Při konstrukci nebylo zraněno žádné zvíře ani osoba. Testování probíhalo ve vnitřních prostorách laboratoře FSv ČVUT.\\
