\chapter{RTI}
\label{4-RTI}

Metoda RTI (Reflectance Transformation Imaging) je výpočetní fotografická metoda, která zachycuje tvar a barvu objektu a následně umožňuje interaktivní nasvícení objektu z jakéhokoli směru pomocí specializovaného softwaru. Metoda umožňuje také matematické zvýraznění povrchu objektu, které vyzdvihne i velmi jemné detaily a díky těmto vlastnostem se využívá například pro zachycení jemných, špatně čitelných rytin, numismatice či v restaurování obrazů. Metoda i specializovaný software jsou volně dostupné.

\section{Princip metody}

Vstupními daty metody je série snímků (RGB) z jednoho neměnného stanoviska a různými směry osvětlení objektu, poloha zdroje světla je známá pro každý snímek.  \\ 

Normálový vektor, je vektor kolmý k povrchu objektu v každém jeho bodě. Velikost úhlu dopadu světelného paprsku na povrch objektu a jeho odrazu je vůči normále stejná (viz obrázek). \\

%doplnit obrázek!!!!

Díky sérii forografií ze stejného stanoviska a s různými směry osvětlení, je možné následně určit normálový vektor pro každý pixel snímku a tím uložit do snímku informaci o třetím rozměru objektu.\\

\section{Postup snímkování}

Fotoaparát je umístěn tak, aby osa záběru byla kolmá na snímaný objekt a následně je pořízena série fotografií z neměnného stanoviska a různými směry osvětlení objektu. Fotoaparát se během snímkování nesmí pohnout, proto je doporučeno jej ovládat dálkově specializovaným SW (např. Nikon Camera Control Pro 2).

Postup snímkování se dělí na dva druhy podle způsobu určení polohy zdroje světla - pomocí světelného dómu, nebo odrazných černých či červených koulí.

\begin{enumerate}
	\item {Odrazné koule}\\
	Při snímání je k zájmovému objektu naistalována černá či červená odrazná koule a to tak, aby nevrhala stín na zájmový objekt při jakémkoli směru osvícení v sérii a zabírala alespoň 250 pixelů z celého snímku. Následně je pohybováno se zdrojem světla (např. externí blesk) po povrchu polokoule tak, aby její pokrytí bylo rovnoměrné - viz obrázek.
	%obrázek
	Při zpracování snímků je na každé fotografii automaticky vyhledána odrazná koule (možno upravit manuálně) a podle odrazu světla na ní je určena poloha zdroje světla.
	
	\item {Světelný dóm}\\
	%obrázek 
	Při využití světelného dómu je poloha světel určena pouze jednou - do středu dómu je umístěna odrazná koule, pořízena série fotografií s osvětlením ze všech světel dómu a následně jsou snímky zpracovány v RTI a vyexportovány soubory s polohou světel. 
	Při snímání zájmovýcj objektů již není třeba znovu určovat polohu světel, pouze se do projektu při zpracování nahraje soubor s již známými polohami světel. Výhoda je v rychlejším pořízení fotografií, můžeme si být jistí, že zroje světla jsou skutečně rozmístěny po polokouli a nechybí data z důvodu znepřístupnění oblasti stativem fotoaparátu.
	
	%obrázek chybějících odrazů kvůli stativu
	
	 Nevýhodou je limitace velikosti objektu.
	
\end{enumerate}

%doplnit obrázek!!!!

\section{Postup zpracování}

\section{Zhodnocení metody}

