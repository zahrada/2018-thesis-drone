\chapter{Úvod}
\label{1-uvod}

Bezpilotní letadla neboli drony jsou rychle vyvýjeným odvětvím v zeměměřičtsví. Nedostatek kvalitní pracovní síli nahrává automatizaci sběru dat, tedy dronům a skenerům.\\
V dnešní době dron plní funkci nosiče fotogrametrické kamery nebo skeneru. Výužívají se tedy pro sběr objemných dat za krátkou dobu. Výstupy z těchto dat jsou ortofota, fotoplány, mračna bodů a z nich 3D modely.\\
V této diplomové práci je popisován stavba drona na otevřené platformě Aurduino, který by mohl nahradit výtyčku při různých zeměměřičských pracích.\\
Pokud by se na drona implementovala GNSS aparatura s RTK, dal by se dron využít pro vytyčování. Po zadání souřadnic uživatel dron by přeletěl na zadané místo a přistál by. Po příchodu uživatele by dron vzletěl, držel by pozici a uživatel by podle laserové stopy stabilizoval bod.\\
Šlo by vytvořit zařízení které by se skládalo z hranolu, podle kterého by dron udržoval polohu nad tímto zařízením. Využití by se našlo v oblastech s vysokými objekty (stromy, budovy), drony s GNSS aparaturou by létal nad vysokými objekty s ideální konfigurací satalitů a uživatel se zařízením by na zemi měřil polohu přes GNSS. Výška drona by se měřila přes laserový dálkoměr.\\
Další možností by bylo upevnění všesměrného hranolu na drona, pro přesné určování polohy drona a následně jeho ovládání přes totální stanici.\\
Pro rozvinutí těchto projektů je potřeba znát daná problematika, proto se diplomová práce zabývá základní stavbou drona.\\