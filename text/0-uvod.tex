\chapter{Úvod}
\label{0-uvod}

Bezpilotní letadla neboli drony jsou rychle vyvýjeným odvětvím v zeměměřičtsví. Nedostatek kvalitní pracovní síli nahrává automatizaci sběru dat, tedy dronům a skenerům.\\
V dnešní době dron plní funkci nosiče fotogrametrické kamery nebo skeneru. Výužívají se tedy pro sběr objemných dat za velmi krátkou dobu. Výsledky po zpracování jsou ortofota, fotoplány, mračna bodů a z nich 3D modely.\\
V této diplomové práci je popisován stavba drona na otevřené platformě Aurduino, který by mohl nahradit výtyčku při různých zeměměřičských pracích.\\
Pokud by se na drona implementovala GNSS aparatura s RTK, dal by se dron využít pro vytyčování. Po zadání souřadnic uživatelem, dron by přeletěl na zadané místo a přistál by. Po příchodu uživatele by dron vzletěl, držel by pozici a uživatel by podle laserové stopy stabilizoval bod.\\
Dále by šlo vytvořit zařízení které by se skládalo z hranolu, podle kterého by dron udržoval polohu nad zařízením. Využití by se našlo v oblastech nepříznivých po GNSS (vysokýmé objekty: stromy, budovy). Dron s GNSS aparaturou by létal nad vysokými objekty s ideální konfigurací satalitů a uživatel se zařízením by na zemi měřil polohu přes GNSS. Výška drona by se měřila přes laserový dálkoměr.\\
Pokud by dron dokázal komunikavat s totální stanicí, získali bychom přesné souřadnice letu drona, které by se dali použít pro přesného definování letu. Využití by se našlo při mapování skal. Nápad propojit totální stanici s dronem napadl doktora Horu z firmy Exact. s.r.o.\\
Pro rozvinutí těchto projektů je potřeba znát problematika letu drona, proto se diplomová práce zabývá základní stavbou drona.\\