\chapter{Úvod}
\label{0-uvod}


Bezpilotní letadla neboli drony jsou rychle vyvíjeným odvětvím v zeměměřičství. Nedostatek kvalitní pracovní síly nahrává automatizaci sběru dat, tedy dronům a skenerům.\\
V dnešní době dron plní funkci nosiče fotogrammetrické kamery nebo skeneru. Využívá se pro sběr objemných dat za velmi krátkou dobu. Výsledky po zpracování jsou ortofota, fotoplány, mračna bodů a z nich 3D modely.\\
V této diplomové práci je popisována stavba dronu na platformě Aurduino, který by mohl nahradit výtyčku při různých zeměměřičských pracích.\\
Pokud by se na dron implementovala GNSS aparatura s podporou RTK, dal by se dron využít pro vytyčování. Po zadání souřadnic uživatelem, dron by přeletěl na zadané místo a přistál by. Po příchodu uživatele by dron vzletěl, držel by pozici zadaných souřadnic a uživatel by, podle laserové stopy, stabilizoval bod.\\
Po připevnění laserového dálkoměru na dron a implementaci automatického cílení dle hranolu drženého uživatelem, by dron držel pozici nad hranolem. Využití by se našlo v nepříznivých oblastech pro GNSS aparatury (vysoké objekty: stromy, budovy). Dron s GNSS aparaturou by létal nad vysokými objekty s ideální konfigurací satelitů a uživatel by s hranolem na zemi měřil polohu přes GNSS na dronu.\\
Pokud by dron dokázal komunikovat s totální stanicí, získali bychom přesné souřadnice letu dronu, které by se dali použít pro přesné definování letové dráhy. Využití by se našlo při měření skal, mostů a sloupů.\\
Pro uskutečnění nápadů je potřeba znát problematika letu dronu, proto se diplomová práce zabývá stavbou drona.\\
V práci se čtenář dozví o teorii letu dronů, použitých komponent pro stavbu a následném sestavování.\\

%V úvodu práce se čtenář dozví něco málo o teorii létání dronů. V následující kapitole jsou uvedeny komponenty a jejich popis. V kapitole Konstrukce jsou prvně vysvětleny dílčí kroky a následně celkový přehled o propojení komponent a popis algoritmu, který ovládá dron. Pro ovládání byla vytvořena aplikace pro mobilní operační systém Android, její uživatelský manuál je popsán v Ovládání.\\